\documentclass{article}

\begin{document}

\title{ON ORGANIZING INTERNATIONAL CONFERENCES}
% PLEASE USE ALL CAPITAL LETTERS FOR THE TITLE

\author{Bradley P. Carlin$^1$ and Antonietta Mira$^2$\\
$^1$ University of Minnesota, USA\\
E--mail: \texttt{carlin@stat.umn.edu}\\
$^2$ University of Insubria, Italy\\
E--mail: \texttt{antonietta.mira@uninsubria.it}\\
}


\date{}

\maketitle

%\bibliographystyle{plain}
% THE REFERENCES BELOW ARE OBTAINED USING THE ``PLAIN' BIBLIOGRAPHY STYLE

Statisticians are increasingly faced with the task of organizing
international conferences: it takes a lot of work (as in
\cite{Mira:Tierney:2001} and \cite{Carlin:Louis:2000}) but it can be
fun (see \cite{Box:1980})!

We hope it will be an enjoyable and interesting experience for
everybody.


\begin{thebibliography}{1}

\bibitem{Box:1980} 
G.E.P. Box. 
\newblock There's no theorem like {B}ayes theorem 
(to the tune of ``{T}here's no business like show business'').
\newblock In {\em Bayesian Statistics}, eds. J.M. Bernardo {\it et
al.}, Valencia, University Press, 12--14, 1980.

\bibitem{Carlin:Louis:2000} B.P. Carlin and T.A. Louis.  \newblock
{\em {B}ayes and Empirical {B}ayes Methods for Data Analysis}.
\newblock Chapman and Hall/CRC Press, second edition, 2000.

\bibitem{Mira:Tierney:2001}
A. Mira and L. Tierney. 
\newblock Efficiency and convergence properties of slice samplers.
\newblock {\em Scandinavian Journal of Statistics}, 29:1--12, 2001.

\end{thebibliography}

\end{document}
